\documentclass[journal,12pt,onecolumn,draftclsnofoot,]{article}
\usepackage[utf8]{inputenc}
\usepackage[brazil]{babel}
\usepackage{indentfirst}
\usepackage{setspace}
\usepackage{amsmath}
\usepackage[dvips]{graphicx}
\usepackage{subfigure}

\begin{document}
	
	\title{Relatório do Projeto eDOE \\ Laboratório de Programação 2}
	\author{Anderson Filipe Clemente Silva \\
			Bruno Roberto Silva de Siqueira \\
			João Pedro Santino Espíndula \\
			Vítor Braga Diniz}
	
	\maketitle
	\newpage
	
	\tableofcontents
	
	\newpage
	
	\section{Design Geral}
	O design do projeto foi arquitetado para minimizar o acoplamento do sistema de forma que foram utilizados vários níveis de abstração. No nível mais alto, há o Mediator, que tem a função de criar, controlar e integrar os controladores ….	
	
	\section{Caso de Uso 1}
	No primeiro caso, deve ser criado um CRUD (\textit{Create}, \textit{Read}, \textit{Update} e \textit{Delete}) de usuários, os quais podem ser doadores ou receptores de itens. Para isso, a equipe de desenvolvimento implementou duas classes: “UsuarioConroller” -- que tem como funções principais criar e administrar conjuntos de usuários -- e “Usuario” -- que é a abstração de um doador ou receptor no sistema.
	
	Em “UsuarioController”,  foi criado um mapa (LinkedHashMap, uma vez que a ordem de cadastro importa) de usuários, cujo identificador único é o CPF ou CNPJ e, também, um set de classes, tal que as classes são a categoria a qual um usuário pode pertencer, como igreja, ONG, sociedade, pessoa física. Além disso, foram implementados os métodos \textit{adicionaDoador()} para cadastrar usuário doador, \textit{atualizaUsuario()} e \textit{removeUsuario()} para atualizar os atributos de um usuário qualquer e remover um usuário, respectivamente.
	
	Já “Usuario” possui id, nome, e-mail, telefone, categoria e uma variável indicadora se o usuário é doador ou receptor. Há, também, um conjunto de métodos get -- para informar às classes externas os valores dos atributos -- e set -- para alterar os atributos.
			
	\section{Caso de Uso 2}
	O caso 2 pede para que os doadores possam inserir os itens a serem doados no sistema. Dessa forma, foram criadas a classe Item (que é a abstração de um item no sistema), a classe abstrata ItemController, a qual tem como objetivo gerenciar todos os itens do sistema e a classe DoadosController, que gerencia os itens doados.
	
	Na classe ItemController, foram implementados 2 atributos, um contador, que contém o identificador (ID) único do item, e um mapa de mapa-- o qual relaciona um Usuário a um outro mapa que, por sua vez, relaciona o id do item ao próprio item --, uma vez que é necessário que o item esteja externamente relacionado ao seu id único e ao seu respectivo usuário. Foram implementados métodos para cadastro, atualização e remoção de itens. Aquele gera o id único incrementando 1 para o próximo item a ser cadastrado e grava os valores necessários no mapa. Além disso, foi incorporado o método \textit{listaTodos()} que lista todos os itens cadastrados em uma ordem específica solicitada pelo usuário. 

	Na classe DoadosController, que extende de ItemController -- afinal, é, também, um controlador de itens, mas específico àqueles doados --, foram implementados um \textit{set} de descrições, indentificando quais as descrições de itens já existentes no sistema, e métodos para cadastrar e exibir item.

	Já na classe Item, foram implementados os seguintes atributos necessários para descrever um item com todas as suas propriedades: ID, descrição, quantidade, tags e usuário. Além disso, há um conjunto de métodos \textit{get} e \textit{set} para transmissão de dados para outras classes.
	
	
	\section{Caso de Uso 3}
	O caso 3 demanda uma ferramenta de pesquisa de itens a serem doados, além da listagem de todos os descritores e dos itens cadastrados em ordem alfabética e em ordem especificada pelo usuário, respectivamente. Com isso, foram implementados os métodos \textit{listaDescritorDeItensParaDoacao()}, \textit{listaItensParaDoacao()} e \textit{pesquisaItemParaDoacaoPorDescricao()} na classe DoadosController, a fim de listar os descritores e itens cadastrados no sistema nas ordens mencionadas e para pesquisar itens para a doação utilizando a descrição como parâmetro de busca.
	
	
	\section{Caso de Uso 4}
		
	
	\section{Caso de Uso 5}
	
	
	\section{Caso de Uso 6}

	
	\section{Caso de Uso 7}	
		
	
	\section{Link para repositório no GitHub}
	https://github.com/pedroespindula/eDoe-LP2.git
	
\end{document}

